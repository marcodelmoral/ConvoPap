% \vspace*{-2.3cm}
% \begin{center}
%     \normalfont\Huge\bfseries{Resumen}
% \end{center}
El cáncer cérvico-uterino es un padecimiento con graves repercusiones en todos
los estratos que componen nuestra sociedad. Es un tipo de cáncer que tiene una tasa
sumamente alta de recuperación, con la condición de que se diagnostique de
manera expedita y precisa. En nuestro país aún mueren muchas mujeres a causa de
este cáncer y, siendo causa directa los problemas en el diagnóstico, se propone
la creación de una herramienta que agilice y profundice este procedimiento. Para
el diagnóstico de cáncer cervicouterino se requiere un análisis profundo de
imágenes tomadas de muestras de las pacientes en un examen llamado Papanicolau.
Se procede a observar detenidamente las laminillas y buscar células atípicas que
se presentan en distintas variedades. En particular se buscan aquellas que
indiquen malformaciones en el núcleo celular, clásico indicador de cambios en el
código genético y, por ende, cáncer. 

Múltiples factores inciden en la eficacia terminal del diagnóstico: desde la
edad del paciente, experiencia del tomador de la muestra, contaminación
ambiental, calidad del microscopio, estado mental del experto cito-tecnólogo,
degradación de la vista, actividad sexual reciente y un sinfín de factores que
crean ruido en la emisión del diagnóstico. Esto sumado a que en nuestro país (y a
nivel mundial), se ha incrementado exponencialmente la toma de muestras, esto
debido a políticas públicas de sanidad aplicadas para combatir el cáncer. 

Se propone en este trabajo el diseño conceptual de una herramienta de
diagnóstico asistido por computadora, capaz de permitir al experto
cito-tecnólogo clasificar, identificar y contar células atípicas dentro de una
muestra de Papanicolau con el objetivo de incrementar la tasa de revisión por
jornada laboral, reducir la cantidad de falsos positivos y falsos negativos,
reducir la fatiga visual que degrada la vista del experto y aumentar la eficacia
total del procedimiento de diagnóstico.

Para ello se pretenden diseñar dos componentes: el primero es un Sistema de
Diagnóstico Asistido por Computadora basado en imágenes que contará como motor
de inferencia una Red Neuronal Convolucionada, que presenta la ventaja de no
requerir una minería de características previas para analizar la imagen, lo cual
simplifica el modelo y permite implementarlo en el segundo componente; un
sistema de hardware embebido, que es una computadora miniaturizada con
restricciones de memoria y procesamiento pero que, si se optimiza previamente el
modelo de clasificación, puede realizar todos los cálculos matriciales necesarios
para la ejecución de una Red Neuronal Convolucionada. \\
La propuesta pretende ofrecer una solución integral y de fácil implementación,
así como bajo costo, para maximizar el horizonte de implementación de la misma.

\textbf{Palabras clave:} Deep learning, Cáncer Cérvico-uterino, Red Neuronal
Convolucionada, Diagnóstico Asistido por Computadora