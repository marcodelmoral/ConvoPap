
\large
\textbf{Breve resumen y lo que expresa esta tesis} \\
% \textnormal{Blablablablablabla}
% En el primer capítulo de esta tesis se analizará el contexto
% del estudio y se presentará una breve introducción del pasado, presente y
% futuro del Instituto Mexicano del Seguro Social. Haciendo énfasis en el
% contexto histórico, social y cultural del mismo en nuestro país. El segundo
% capítulo tocaremos el planteamiento del problema. Se darán las
% justificaciones, objetivos, hipótesis y demás temas tocantes a la
% delimitación metodológica del estudio realizado. La metodología se mostrará
% en el tercer capítulo, donde se desglosarán todas las ciencias y sus
% técnicas que asistirán en la creación de la solución propuesta. La
% realización total de la propuesta se explicará en el quinto capítulo. Aquí
% se describirá la invención del nuevo modelo de diagnóstico asistido por
% inteligencia artificial. El sexto capítulo nos mostrará los esfuerzos
% realizados para la invención del dispositivo de hardware para diagnóstico de
% cáncer cérvicouterino. Se analizará desde el hardware hasta el software para
% la construcción del mismo.
