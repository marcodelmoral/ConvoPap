    % El cáncer cérvico-uterino (CCU) \nomenclature{CCU}{cáncer cérvico-uterino}
    El cáncer cérvico-uterino (\hyperlink{abbr}{CCU}\nomenclature{CCU}{Cáncer
    Cérvico-uterino}) es una patología apremiante en los países en vías de
    desarrollo como México. Disminuir su incidencia en la población deber una de
    las prioridades de cualquier institución en pos del bienestar humano. El
    flagelo de esta enfermedad en nuestro país no solo aflige en manera de
    salud, sino que permea en el tejido social convirtiéndose en un problema de
    salud pública.~\cite{INEGI2018} \\
    Problemas en el diagnóstico inciden directamente en la tasa de casos que se
    presentan en el país. Dichos problemas no se remiten a la habilidad del
    médico como tal; él, como profesional, está en el ápice de sus capacidades y
    trabaja jornadas sobrehumanas para poder suplir la necesidad de
    diagnósticos. Es precisamente en esta cuestión donde radica el núcleo del
    problema, el profesional médico no se da abasto para diagnosticar todas las
    muestras que se recaban en nuestros laboratorios.~\cite{Ahlan2014} \\    
    La solución que se propone es traer las técnicas más vanguardistas
    existentes en materia de Inteligencia Artificial
    (\hyperlink{abbr}{IA}\nomenclature{IA}{Inteligencia Artificial}),
    Tecnologías de la Información
    (\hyperlink{abbr}{TI}\nomenclature{TI}{Tecnologías de la Información}) y
    Sistemas Embebidos
    (\hyperlink{abbr}{SE}\nomenclature{SE}{\nomenclature{SE}{Sistema embebido}})
    , para ponerlas al servicio de los profesionales de la salud para potenciar
    su capacidad de diagnóstico; haciéndolos más rápidos y más precisos en la
    tarea. El grado de adopción de estos conceptos es exponencial es
    instituciones privadas y públicas de primer mundo y permiten sobrecargar la
    habilidad de profesionales en tantos rubros como inventiva se
    tenga.~\cite{Bhaskar2016}\\
    Es en esta era donde la adopción masiva de máquinas para asistir al hombre
    mostró su gran capacidad de cambio para la sociedad global. La segunda de
    estas revoluciones, la digital, catapultó el desarrollo y avance científico
    a magnitudes jamás antes vistas en la historia humana. Por ello, es
    imperativo para el Ingeniero Industrial, no solo aplicar las técnicas y
    métodos más avanzados para resolver problemas enormes en la industria, sino
    que también es crucial que aporte su conocimiento a dominios insospechados
    por el paradigma del gremio. Sus capacidades y conocimientos son una gran
    adición a cualquier institución y la utilidad de estos ingenieros solo está
    limitada por la creatividad para resolver problemas del ingeniero
    mismo.~\cite{Kelleher2015} \\ 
    Se hará un diseño conceptual de una solución que se manifestará en forma de
    un sistema de software capaz de analizar imágenes de exámenes de Papanicolau
    (\hyperlink{abbr}{PAP}\nomenclature{PAP}{Papanicolau}) y, mediante la
    técnica de inteligencia artificial llamada Deep Learning
    (\hyperlink{abbr}{DL}\nomenclature{DL}{Deep Learning}) o Aprendizaje
    Profundo, detectará si la muestra tiene incidencia de cáncer o no, así como
    clasificar el grado de lesión en el que se encuentra cada célula. Así mismo,
    se pretende encapsular este modelo matemático en un dispositivo de hardware
    capaz de acoplarse a un microscopio y realizar, en una sola actividad, la
    captura de imagen y la realización del diagnóstico médico de tal
    muestra.~\cite{Engelbrecht2005} \\

