% Creo que ya quedó el planteamiento :D
El \hyperlink{abbr}{CCU} es uno de los dos tipos de cáncer ubicuos al sexo
femenino. Siendo, por su grado de mortandad en comparación a todos los demás
tipos, el segundo de mayor importancia \footnote{El primero es el cáncer de
mama}. A nivel global,  el 12\% de todos los cánceres femeninos son
CCU.~\cite{CancerToday-InstitutionalAgencyforResearchonCancerWHO2018}

Aparte de la tasa de mortandad, el \hyperlink{abbr}{CCU} resulta ser un factor
de alto impacto social por varias circunstancias. La primera es que, por la
naturaleza del tratamiento, la persona queda inhabilitada por la agresividad del
mismo; en familias de escasos recursos, donde la probabilidad de perecer de
cáncer es más alta y donde en varias ocasiones la madre es la única que trae
sustento al hogar. En segundo lugar, si la persona cuenta con acceso a salud
pública, tratar y curar el cáncer conlleva en una gran carga económica para el
Estado y para los contribuyentes; si la persona se decide por la salud privada,
el costo de la misma fácilmente puede hacer que baje una o varias clases
económicas.~\cite{SecretariadeSalud2015a}

La única manera para reducir estos impactos es mediante un diagnóstico rápido y
temprano del cáncer: la tasa de supervivencia es directamente proporcional a que
tan temprano se haga el diagnóstico y el cáncer en etapas tempranas,
lógicamente, es más rápido y menos costoso de curar.~\cite{WorldHealthOrganization}

Este diagnóstico se realiza mediante el análisis visual de las muestras tomadas
por el examen de Papanicolau. Dicho examen, al ser considerado de rutina y con
carácter crucial en los hospitales genera un volumen de pruebas bastante
elevado, lo que conlleva a una reducción en la eficacia de diagnóstico que puede
costar vidas e incide directamente en la precisión. La eficacia se reduce debido
a que muchas veces los hospitales o clínicas carecen del personal necesario para
analizar todas las muestras, por ejemplo, un laboratorio consultado de patología
recibe un número de muestras que rondan las 500 unidades diarias, lo cual
resulta excesivo para las personas que trabajan en dicho laboratorio\footnote{Se
entrevistó a un laboratorio en un hospital con solo dos cito-tecnólogos para
servir 5 municipios}.~\cite{DelMoral2017} 

Las pruebas manuales de \hyperlink{abbr}{PAP} son un reto debido a que es un
trabajo repetitivo, intensivo, tiene alta incidencia de falsos positivos y
negativos, conlleva riesgo ergonómico y puede ser afectado significativamente
por el ambiente en el cual se tomó la muestra. Esta dificultad aparte se ve
exacerbada por la falta de cito-tecnólogos.~\cite{Cucoranu2014}

La precisión se reduce cuando una persona tiene que analizar demasiadas muestras
y por fatiga ocular o mental, comienza a generar errores en el diagnóstico, es
común que una persona que se dedique a realizar el diagnóstico realice horas
extras y tenga sobrecarga de trabajo. Otro factor que afecta en la eficacia del
diagnóstico, es que cada experto tiene su propia heurística mental al
diagnosticar casos limítrofes, esto genera subjetividad y falta de consenso
médico en el diagnóstico con lo cual se genera lentitud en el mismo. Todo esto
compone la problemática técnica. Los especialistas, requieren herramientas que
los asistan en la realización de su trabajo para poder incrementar el volumen de
muestras a analizar, reducir riesgos de falsos positivos y negativos, mejorar la
ergonomía del trabajo y que, en particular, asista al especialista dentro del
laboratorio a identificar células que presenten lesiones cancerígenas. Esta
herramienta debe ser portátil, con alto grado de usabilidad y basada en
tecnologías tanto probadas como de bajo costo.~\cite{Lugo-Reyes2014}
