% Poner los abstract, creo que hay que volverlo a hacer

Existen aplicaciones tradicionales de \hyperlink{abbr}{RNA} y Aprendizaje
Automático o Machine Learning (\hyperlink{abbr}{ML}\nomenclature{ML}{Machine
Learning}) en los temas relacionados al diagnóstico y detección de
\hyperlink{abbr}{CCU}. Inclusive algunos realizan el análisis de sus técnicas en
países en vías de desarrollo con situaciones similares a
México.~\cite{HussainWanandIshakWanandFadzilah1994}

Dentro del área de cito-patología, la mayoría de las aplicaciones de RNA han
sido en el área de cito-patología en general, no solo para la detección de
cáncer, por lo tanto se tienen precedentes de una buena interacción entre la
técnica propuesta y el problema a solucionar. Por lo cual podemos decir que las
herramientas de asistencia al diagnóstico son ubicuas al dominio
analizado.~\cite{Pouliakis2016}

También se ha encontrado que distintas técnicas de \hyperlink{abbr}{DL} mejoran
la eficiencia del diagnóstico y permiten reducir la complejidad de los datos de
entrenamiento y también, las ConvNets permiten la reducción considerable de
falsos positivos y falsos negativos en el análisis de
imágenes.~\cite{Khosravi2017}

En el área específica de la búsqueda de cáncer cérvico-uterino, se ha
encontrado que distintos algoritmos pueden alcanzar grados de asertividad lo
suficientemente buenos. Lo cual es indicación de que el problema puede ser
resuelto con la técnica propuesta. Si tomamos en cuenta el método de toma de la
muestra, también encontramos uso de \hyperlink{abbr}{RNA} para  la
clasificación de las mismas.~\cite{Ampazis2004}

Implementar técnicas como Transfer Learning
(\hyperlink{abbr}{TL}\nomenclature{TL}{Transfer Learning}) están permitiendo
precisiones superiores a las antes alcanzadas, mediante la particularización de
modelos capaces de clasificar cientos de clases distintas y ajustar ese
entrenamiento al diagnóstico
médico.~\cite{Sarwar2013}\cite{Sarwar2015}\cite{Zhang2017}

\begin{enumerate}[label=\textbf{\arabic*}]
    \item \textbf{Artificial Intelligence in Medical Application: An Exploration}~\cite{HussainWanandIshakWanandFadzilah1994}
    \begin{itemize} 
        \item{\textbf{Autores:}} Wan Hussain Wan Ishak, Fadzilah Siraj
        \item{\textbf{Abstract:}} \textit{The advancement in computer technology has
        encouraged the researchers to develop software for assisting doctors in
        making decision without consulting the specialists directly. The
        software development exploits the potential of human intelligence such
        as reasoning, making decision, learning (by experiencing) and many
        others. Artificial intelligence is not a new concept, yet it has been
        accepted as a new technology in computer science. It has been applied in
        many areas such as education, business, medical and manufacturing. This
        paper explores the potential of artificial intelligence techniques
        particularly for web-based medical applications. In addition, a model
        for web-based medical diagnosis and prediction is proposed. }
        \item{\textbf{Problemática:}} Los países en vías de desarrollo carecen
        de médicos suficientes por lo que la mortandad de muchas enfermedades es
        mayor en tales países
        \item{\textbf{Técnicas:}}  \hyperlink{abbr}{IA}, RNA,
        ingeniería de software, lógica difusa
        \item{\textbf{Aporte:}} Sistema web para diagnóstico y predicción
    \end{itemize}
    \item \textbf{Artificial Neural Networks as Decision Support Tools in Cytopathology: Past, Present, and Future}~\cite{Pouliakis2016}
    \begin{itemize} 
        \item{\textbf{Autores:}} Pouliakis, A Karakitsou, E Margari, N Bountris,
        P Haritou, M Panayiotides, J Koutsouris, D Karakitsos, P
        \item{\textbf{Abstract:}} \textit{OBJECTIVE: This study aims to analyze the role
        of artificial neural networks (ANNs) in cytopathology. More
        specifically, it aims to highlight the importance of employing ANNs in
        existing and future applications and in identifying unexplored or poorly
        explored research topics. STUDY DESIGN: A systematic search was
        conducted in scientific databases for articles related to cytopathology
        and ANNs with respect to anatomical places of the human body where
        cytopathology is performed. For each anatomic system/organ, the major
        outcomes described in the scientific literature are presented and the
        most important aspects are highlighted. RESULTS: The vast majority of
        ANN applications are related to cervical cytopathology, specifically for
        the ANN-based, semiautomated commercial diagnostic system PAPNET. For
        cervical cytopathology, there is a plethora of studies relevant to the
        diagnostic accuracy; in addition, there are also efforts evaluating
        cost-effectiveness and applications on primary, secondary, or hybrid
        screening. For the rest of the anatomical sites, such as the
        gastrointestinal system, thyroid gland, urinary tract, and breast, there
        are significantly less efforts relevant to the application of ANNs.
        Additionally, there are still anatomical systems for which ANNs have
        never been applied on their cytological material. CONCLUSIONS:
        Cytopathology is an ideal discipline to apply ANNs. In general,
        diagnosis is performed by experts via the light microscope. However,
        this approach introduces subjectivity, because this is not a universal
        and objective measurement process. This has resulted in the existence of
        a gray zone between normal and pathological cases. From the analysis of
        related articles, it is obvious that there is a need to perform more
        thorough analyses, using extensive number of cases and particularly for
        the non explored organs. Efforts to apply such systems within the
        laboratory test environment are required for future uptake.}
        \item{\textbf{Problemática:}} Analizar el rol de la técnica RNA en la
        cito-patología
        \item{\textbf{Técnicas:}}  \hyperlink{abbr}{IA}, RNA
        \item{\textbf{Aporte:}} Se determinó que la mayoría de los usos de redes
        neuronales para cito-patología son en el área cervicouterina
    \end{itemize}
    \item \textbf{Using Deep Learning to enhance cancer diagnosis and classification}~\cite{Fakoor2013}
    \begin{itemize} 
        \item{\textbf{Autores:}} Fakoor, Rasool Ladhak, Faisal Nazi, Azade
        Huber, Manfred
        \item{\textbf{Abstract:}} \textit{Using automated computer tools and in particular machine learning to facilitate and enhance medical analysis and diagnosis is a promising and important area. In this paper, we show that how unsupervised feature learn-ing can be used for cancer detection and can-cer type analysis from gene expression data. The main advantage of the proposed method over previous cancer detection approaches is the possibility of applying data from various types of cancer to automatically form fea-tures which help to enhance the detection and diagnosis of a specific one. The technique is here applied to the detection and classifica-tion of cancer types based on gene expression data. In this domain we show that the per-formance of this method is better than that of previous methods, therefore promising a more comprehensive and generic approach for cancer detection and diagnosis.}
        \item{\textbf{Problemática:}} Se require mejorar la precisión de los
        sistemas de diagnóstico de cáncer basados en técnicas de  \hyperlink{abbr}{IA}
        \item{\textbf{Técnicas:}} DL, ConvNets, Autoencoders, ML, Principal Component analysis
        \item{\textbf{Aporte:}} Se mejoró la eficiencia del diagnóstico y se
        logró reducir la complejidad de los datos
    \end{itemize}
    \item \textbf{Pap-Smear Classification Using Efficient Second Order Neural Network Training Algorithms}~\cite{Ampazis2004}
    \begin{itemize} 
        \item{\textbf{Autores:}} Ampazis, Nikolaos Dounias, George Jantzen, Jan
        \item{\textbf{Abstract:}} \textit{In this paper we make use of two highly efficient second
         or- der neural network training algorithms, namely the LMAM (Levenberg- Marquardt 
         with Adaptive Momentum) and OLMAM (Optimized Levenberg-Marquardt with Adaptive Momentum), 
         for the construction of an efficient pap-smear test classifier. The algorithms are methodologically 
         similar, and are based on iterations of the form employed in the Levenberg-Marquardt (LM) 
         method for non-linear least squares problems with the inclusion of an
         additional adaptive momentum term 
         arising from the formulation of the training task as a constrained optimization problem. The classification 
         results obtained from the application of the algorithms on a standard benchmark pap-smear data set reveal 
         the power of the two methods to obtain excellent solutions in difficult classification problems whereas other
          standard computational intelligence techniques achieve inferior performances.}
        \item{\textbf{Problemática:}} Detectar CCU mediante el análisis de
        citología PAP.
        \item{\textbf{Técnicas:}} Algoritmos de entrenamiento, ANFIS, NeuroFuzzy
        \item{\textbf{Aporte:}} Se logró detectar displasia mediante la
        combinación de varias técnicas de  \hyperlink{abbr}{IA}
    \end{itemize}
    \item \textbf{Deep Learning in Medical Imaging: General Overview}~\cite{Lee2017}
    \begin{itemize} 
        \item{\textbf{Autores:}} Lee, June-Goo Jun, Sanghoon Cho, Young-Won Lee,
        Hyunna Kim, Guk Bae Seo, Joon Beom Kim, Namkug
        \item{\textbf{Abstract:}} \textit{The artificial neural network (ANN)-a machine 
        learning technique inspired by the human neuronal synapse system-was introduced in
         the 1950s. However, the ANN was previously limited in its ability to solve actual 
         problems, due to the vanishing gradient and overfitting problems with training of de
         ep architecture, lack of computing power, and primarily the absence of sufficient data
          to train the computer system. Interest in this concept has lately resurfaced, due to 
          the availability of big data, enhanced computing power with the current graphics processing
           units, and novel algorithms to train the deep neural network. Recent studies on this technology
           suggest its potentially to perform better than humans in some visual and auditory recognition 
           tasks, which may portend its applications in medicine and healthcare, especially in medical
            imaging, in the foreseeable future. This review article offers perspectives on the history, 
            development, and applications of deep learning technology, particularly regarding its 
            applications in medical imaging.}
        \item{\textbf{Problemática:}} Mejorar la precisión de los algoritmos de
        ML para diagnóstico médico mediante imágenes
        \item{\textbf{Técnicas:}} DL, ML
        \item{\textbf{Aporte:}} El uso de DL en el diagnóstico médico
        puede escalar fácilmente y tiene resultados superiores a los métodos previos.
    \end{itemize}
    \item \textbf{Intelligent Screening Systems for Cervical Cancer}~\cite{Jusman2014}
    \begin{itemize} 
        \item{\textbf{Autores:}} Yessi Jusman, Siew Cheok Ng, and Noor Azuan Abu
        Osman
        \item{\textbf{Abstract:}} \textit{Advent of medical image digitalization leads to 
        image processing and computer-aided diagnosis systems in numerous clinical applications.
         These technologies could be used to automatically diagnose patient or serve as second opinion
          to pathologists. This paper briefly reviews cervical screening techniques, advantages, and disadvantages. 
          The digital data of the screening techniques are used as data for the computer screening system as replaced 
          in the expert analysis. Four stages of the computer system are enhancement, features extraction, feature selection,
           and classification reviewed in detail. The computer system based on cytology data and electromagnetic spectra data
            achieved better accuracy than other data.}
        \item{\textbf{Problemática:}} Analizar los algoritmos inteligentes
        para diagnosticar CCU mediante software
        \item{\textbf{Técnicas:}} ML
        \item{\textbf{Aporte:}} Sea el algoritmo que se use, se pueden alcanzar
        altos niveles de asertividad y rapidez
    \end{itemize}
    \item \textbf{Hybrid ensemble learning technique for screening of cervical cancer using Papanicolaou smear image analysis}~\cite{Sarwar2015}
    \begin{itemize} 
        \item{\textbf{Autores:}} Sarwar, Abid Sharma, Vinod Gupta, Rajeev
        \item{\textbf{Abstract:}} \textit{OBJECTIVE This paper presents an innovative 
        idea of applying a hybrid ensemble technique i.e. ensemble of ensemble methods 
        for improving the predictive performance of Artificial intelligence based system 
        for screening of cervical cancer by characterization and classification of Pap 
        smear images. METHODOLOGY Papanicolaou smear (also referred to as Pap smear) 
        is a microscopic examination of samples of human cells scraped from the lower, 
        narrow part of the uterus, called the cervix. A sample of cells after being stained
        by using Papanicolaou method is analyzed under microscope for the presence of any unusual 
        developments indicating any precancerous and potentially precancerous changes. 
        Abnormal findings, if observed are subjected to further precise diagnostic subroutines. 
        Examining the cell images for abnormalities in the cervix provides grounds for provision
        of prompt action and thus reducing incidence and deaths from cervical cancer. 
        It is the most popular technique used for screening of cervical cancer. Pap smear test, 
        if done with a regular screening programs and proper follow-up, can reduce cervical cancer 
        mortality by up to 80\%. The contribution of this paper is that we have pioneered to apply 
        hybrid ensemble technique to screen cervical cancer by classification of Pap smear data. The 
        hybrid ensemble designed in this work has also presented an idea to use an ensemble of ensemble 
        techniques. Using such a technique, the classification potentials of individual algorithms are fused
        together to gain greater classification accuracy. In addition to this we have also presented a 
        comparative analysis of various artificial intelligence based algorithms for screening of cervical cancer.
        RESULTS The results indicate that hybrid ensemble technique is an efficient 
        method for classification of Pap smear images and hence can be effectively used for
        diagnosis of cervical cancer. Among all the algorithms implemented, the hybrid ensemble
        approach outperformed and expressed an efficiency of about 96\% for 2-class problem and 
        about 78\% for 7-class problem. The results when compared with the all the standalone 
        classifiers were significantly better for both 2-class and 7-class problems.}
        \item{\textbf{Problemática:}} Diagnosticar CCU con análisis de
        imágenes mediante  \hyperlink{abbr}{IA}
        \item{\textbf{Técnicas:}} Ensamble de modelos
        \item{\textbf{Aporte:}} Utilizar métodos combinados eleva la precisión
        arriba del 90\%
    \end{itemize}
    \item \textbf{Artificial Intelligence Based Semi-automated Screening of Cervical Cancer using a Primary Training Database}~\cite{Sarwar2016}
    \begin{itemize} 
        \item{\textbf{Autores:}} Sarwar, Abid
        \item{\textbf{Abstract:}} \textit{Objective: The primary objective of
        this research work is to develop a novel benchmark database of digitized
        and calibrated, cervical cells obtained from slides of Papanicolaou
        smear test, which is done for screening of cervical cancer. This
        database can serve as a potential tool for designing, developing,
        training, testing and validating various Artificial intelligence based
        systems for prognosis of cervical cancer by characterization and
        classification of Papanicolaou smear images. The database can also be
        used by other researchers for comparative analysis of working
        efficiencies of various machine learning and image processing
        algorithms. The database can be obtained by sending a request to the
        corresponding author. Besides developing a rich machine learning
        database we have also presented a novel artificial intelligence based
        hybrid ensemble technique for efficient screening of cervical cancer by
        automated analysis of Papanicolaou smear images. Methodology: The
        correct and timely diagnosis of cervical cancer is one of the major
        problems in the medical world. From the literature it has been found
        that different pattern recognition techniques can help them to improve
        in this domain. Papanicolaou smear (also referred to as Pap smear) is a
        microscopic examination of samples of human cells scraped from the
        lower, narrow part of the uterus, called cervix. A sample of cells after
        being stained by using Papanicolaou method is analyzed under microscope
        for the presence of any unusual developments indicating any precancerous
        and potentially precancerous developments. Abnormal findings, if
        observed are subjected to further precise diagnostic subroutines.
        Examining the cell images for abnormalities in the cervix provides
        grounds for provision of prompt action and thus reducing incidence and
        deaths from cervical cancer. It is the most popular technique used for
        screening of cervical cancer. Pap smear test, if done with a regular
        screening programs and proper follow-up, can reduce cervical cancer
        mortality by up to 80\%. The contribution of this paper is that we have
        created a rich machine learning database of quantitatively profiled and
        calibrated cervical cells obtained from Pap- smear test slides. The
        database so created consists of data of about 200 clinical cases (8091
        cervical cells), which have been obtained from multiple health care
        centers, so as to ensure diversity in data. The slides were processed
        using a multi-headed digital microscope and images of cervical cells
        were obtained, which were passed through various data preprocessing
        subroutines. After preprocessing the cells were morphologically profiled
        and scaled to obtain separate quantitative measurements of various
        features of cytoplasm and nucleus respectively. The cells in the
        database were carefully classified in different corresponding classes
        according to latest 2001-Bethesda system of classification, by
        technicians. In addition to this, we have also pioneered to apply a
        novel hybrid ensemble system to this database in order to evaluate the
        effectiveness of both novel database and novel hybrid ensemble technique
        to screen cervical cancer by categorization of Pap smear data. The paper
        also presents a comparative analysis of multiple artificial intelligence
        based classification algorithms for prognosis of cervical cancer.
        Results: For evaluating the effectiveness and correctness of the digital
        database prepared in this work, authors implemented this database for
        training, testing and validating fifteen different artificial
        intelligence based machine learning algorithms. All algorithms trained
        with this database presented commendable efficiency in screening of
        cervical cancer. For two-class problem all the algorithms trained with
        the digital database showed the efficiencies in range of about 93-95\%
        while as in case of multi class problem algorithms expressed the
        efficiencies in the range of about 69-78\%. The results indicate that
        the novel digital database prepared in this work can be efficiently used
        for developing new machine learning based techniques for automated
        screening of cervical cancer. The results also indicate that hybrid
        ensemble technique is an efficient method for classification of
        pap-smear images and hence can be effectively used for diagnosis of
        cervical cancer. Among all the algorithms implemented, the hybrid
        ensemble approach outperformed and expressed an efficiency of about 98\%
        for 2-class problem and about 86\% for 7-class problem. The results when
        compared with the all the standalone classifiers were significantly
        better for both two- class and multi-class problems.}
        \item{\textbf{Problemática:}} Desarrollar una base de datos novedosa
        para calibrar y probar algoritmos de inteligencia artificial dedicados
        al diagnóstico de CC mediante el análisis de imágenes citológicas de pap
        \item{\textbf{Técnicas:}} Minería de datos
        \item{\textbf{Aporte:}} La base de datos es capaz de entrenar bastantes
        algoritmos distintos con precisión arriba del 90\% sin overfitting
    \end{itemize}
    \item \textbf{Transfer Learning with Partial Observability Applied to Cervical Cancer Screening}~\cite{Fernandes2017}
    \begin{itemize} 
        \item{\textbf{Autores:}} Fernandes, Kelwin Cardoso, Jamie Fernandes,
        Jessica        
        \item{\textbf{Abstract:}} \textit{Cervical cancer remains a significant cause of mortality in low-income countries. 
        As in many other diseases, the existence of several screening/diagnosis methods and subjective physician preferences 
        creates a complex ecosystem for automated methods. In order to diminish the amount of labeled data from each 
        modality/expert we propose a regularization-based transfer learning strategy that encourages source and target models 
        to share the same coefficient signs. We instantiated the proposed framework to predict cross-modality individual risk 
        and cross-expert subjective quality assessment of colposcopic images for different modalities. Thus, we are able to transfer 
        knowledge gained from one one expert/modality to another.}
        \item{\textbf{Problemática:}} La existencia de varios métodos y la
        subjetividad de los expertos reducen la eficacia del diagnóstico de CC
        \item{\textbf{Técnicas:}} TL
        \item{\textbf{Aporte:}} Se mejoró la precisión de los algoritmos
        inclusive utilizando modelos previamente entrenados para otro tipo de clasificación
    \end{itemize}
    \item \textbf{Efficient False Positive Reduction in Computer-Aided Detection Using Convolutional Neural Networks and Random View Aggregation}~\cite{Lu}
    \begin{itemize} 
        \item{\textbf{Autores:}} Holger R. Roth, Le Lu, Jiamin Liu, Jianhua Yao,
        Ari Seff, Kevin Cherry, Lauren Kim and Ronald M. Summers
        \item{\textbf{Abstract:}} \textit{In clinical practice and medical
        imaging research, automated computer- aided detection (CADe) is an
        important tool. While many methods can achieve high sensitivities, they
        typically suffer from high false positives (FP) per patient. In this
        study, we describe a two-stage coarse-to-fine approach using CADe
        candidate generation systems that operate at high sensitivity rates
        (close to 100\% recall). In a second stage, we reduce false positive
        numbers using state-of-the-art machine learn- ing methods, namely deep
        convolutional neural networks (ConvNet). The ConvNets are trained to
        differentiate hard false positives from true-positives utilizing a set
        of2D (two-dimensional) or 2.5D re-sampled views comprising random
        translations, rotations, and multi-scale observations around a
        candidate’s center coordinate. During the test phase, we apply the
        ConvNets on unseen patient data and aggregate all probability scores
        for lesions (or pathology). We found that this second stage is a highly
        selective classifier that is able to reject difficult false positives
        while retaining good sensitivity rates. The method was evaluated on
        three data sets (sclerotic metastases, lymph nodes, colonic polyps) with
        varying numbers patients (59, 176, and 1,186, respectively). Experiments
        show that the method is able to generalize to different applications and
        increasing data set sizes. Marked improvements are observed in all
        cases: sensitivities increased from 57 to 70\%, from 43 to 77\% and from
        58 to 75\% for sclerotic metastases, lymph nodes and colonic polyps,
        respectively, at low FP rates per patient (3 FPs/patient).}
        \item{\textbf{Problemática:}} Se encuentran muchos falsos positivos en
        los diagnósticos realizados mediante técnicas de ML
        \item{\textbf{Técnicas:}} ConvNets
        \item{\textbf{Aporte:}} El uso de DL en 20\% la
        sensibilidad de los sistemas de diagnóstico.
    \end{itemize}
    \item \textbf{DeepPap: Deep Convolutional Networks for Cervical Cell Classification}~\cite{Zhang2017}
    \begin{itemize} 
        \item{\textbf{Autores:}} Ling Zhang, Le Lu, Senior Member, IEEE,
        Isabella Nogues, Ronald M. Summers, Shaoxiong Liu, and Jianhua Yao
        \item{\textbf{Abstract:}} \textit{Automation-assisted cervical screening via Pap smear or liquid-based cytology (LBC) is a highly effective cell imaging based cancer detection tool, where cells are partitioned into "abnormal" and "normal" categories. However, the success of most traditional classification methods relies on the presence of accurate cell segmentations. Despite sixty years of research in this field, accurate segmentation remains a challenge in the presence of cell clusters and pathologies. Moreover, previous classification methods are only built upon the extraction of hand-crafted features, such as morphology and texture. This paper addresses these limitations by proposing a method to directly classify cervical cells without prior segmentation based on deep features, using convolutional neural networks (ConvNets). First, the ConvNet is pretrained on a natural image dataset. It is subsequently finetuned on a cervical cell dataset consisting of adaptively resampled image patches coarsely centered on the nuclei. In the testing phase, aggregation is used to average the prediction scores of a similar set of image patches. The proposed method is evaluated on both Pap smear and LBC datasets. Results show that our method outperforms previous algorithms in classification accuracy (98.3\%), area under the curve (AUC) (0.99) values, and especially specificity (98.3\%), when applied to the Herlev benchmark Pap smear dataset and evaluated using five fold cross-validation. Similar superior performances are also achieved on the HEMLBC (H\&E stained manual LBC) dataset. Our method is promising for the development of automation-assisted reading systems in primary cervical screening.}
        \item{\textbf{Problemática:}} Las aplicaciones anteriores de análisis
        requieren una segmentación muy precisa de la célula
        \item{\textbf{Técnicas:}} DL, ConvNets
        \item{\textbf{Aporte:}} Las ConvNets aplicadas con TL generan un
        rendimiento muy superior inclusive cuando hay incertidumbre en la
        segmentación citológica
    \end{itemize}
\end{enumerate}

