Para solucionar este conjunto de problemas, se requiere algo capaz de
incrementar la tasa de diagnósticos realizados por tiempo y por experto. Así
mismo, este algo debe mantener niveles de asertividad sumamente altos y ser
capaz de emitir decisiones en situaciones donde el consenso médico podría ser
ambiguo.~\cite{Meza-Palacios2017}

Los sistemas de software, en concreto, los \hyperlink{abbr}{SDAC}, son
utilizados en situaciones donde se requiere cumplir los requerimientos antes
mencionados. Al ser digitalizados, nos permitirán procesar grandes volúmenes de
información que de otra manera serían inmanejables y, al incluir en su
funcionamiento técnicas de \hyperlink{abbr}{IA} y \hyperlink{abbr}{DL}, podremos
reducir la subjetividad para llegar a una alta precisión en el diagnóstico y dar
paso a mejoras en la calidad de vida del experto gracias al uso de la
tecnología.~\cite{DominguezHernandez2013} 

Para este software, el motor principal de diagnóstico será la técnica de
\hyperlink{abbr}{IA} llamada Red Neuronal Artificial
(\hyperlink{abbr}{RNA}\nomenclature{RNA}{Red Neuronal Artificial}) en su
variante de Red Neuronal Convolucionada
(\hyperlink{abbr}{ConvNet}\nomenclature{ConvNet}{Red Neuronal Convolucionada}).
En previas implementaciones de \hyperlink{abbr}{SDAC}s para diagnóstico de
cáncer mediante análisis de imágenes citológicas, primero se realizaba un pre
procesamiento en donde, mediante algoritmos de \hyperlink{abbr}{PDI} y dos
etapas exhaustivas y complejas llamadas ingeniería y extracción de
características, se identificaban aquellas que se consideran criterios de
decisión para realizar el diagnóstico y se alimentaban al modelo de
\hyperlink{abbr}{IA}. Esto había sido el proceso habitual de diagnóstico médico
de cáncer mediante software.~\cite{Ashok2016} 

Se propone el uso de las \hyperlink{abbr}{ConvNet}s debido a que, por su
arquitectura, pueden recibir como entrada la imagen completa en lugar de los
criterios de diagnóstico minados en el pre procesamiento y representados de
forma numérica o categórica. Ello reduce la complejidad del sistema y lo hace
menos propenso a fallos. Las \hyperlink{abbr}{RNA}s solamente requieren gran uso
computacional en la fase de entrenamiento, por ello desplegarlas en condiciones
con recursos computacionales bajos es un reto pero es alcanzable, previo
entrenamiento y optimización del modelo.~\cite{Lee2017} 

También, las \hyperlink{abbr}{ConvNet}s alcanzan la mayor precisión hasta ahora
en clasificación de imágenes, siendo el algoritmo más avanzado e implementado en
esta área por ende, es una técnica sumamente fiable y robusta. Esta
característica también nos permite realizar el análisis de imágenes que en otras
metodologías no resultaban idóneas por sus propiedades
técnicas.~\cite{Litjens2017}

Este modelo se desplegará en un dispositivo de hardware basado en
\hyperlink{abbr}{SE}s, en el cual se instalará el software que permitirá
capturar, procesar y clasificar las imágenes y que proveerá de la interface para
interactuar con el experto; esto constituye los dos componentes del
\hyperlink{abbr}{SDAC}, el tendrá que cumplir los siguientes puntos:

\begin{itemize}
    \item{\textbf{Alto grado de usabilidad:}} Toda tecnología genera un grado de
    estrés intrínseco al uso de la misma; al implementar criterios de Usabilidad
    y Experiencia de Usuario o User eXperience
    (\hyperlink{abbr}{UX}\nomenclature{UX}{User eXperience}) se pretende reducir
    dicho estrés al mínimo y así maximizar el uso de la plataforma. Esta debe
    asistir al experto en tomar mejores decisiones, no al
    contrario.~\cite{Nielsen1993}
    \item{\textbf{Código abierto:}}  La plataforma debe estar escrita en su
    totalidad en código abierto y no depender de licencias de software caras y
    que reduzcan la posibilidad de uso de la plataforma. Ello en contraste con
    otros sistemas de diagnóstico de \hyperlink{abbr}{CCU} mediante
    \hyperlink{abbr}{IA} como PAPNET, plataformas como MATLAB o servicios como
    Watson.~\cite{Zhou}
    % \item{Alta compatibilidad}: Debido a la imposibilidad de sincronizar las
    % características técnicas de cada equipo de cómputo que existe en cada
    % hospital o clínica, se requiere que el software sea compatible con gran
    % variedad de plataformas. 
    % \item{Bajos recursos computacionales:} Esta plataforma debe poder ser
    % implementada en computadoras que la mayoría del tiempo no tienen recursos
    % computacionales suficientes, por lo que se debe de tener especial cuidado en el rendimiento.
    \item{\textbf{Fácil mantenimiento:}}  Por la claridad y transparencia del
    lenguaje propuesto (Python), la plataforma será de fácil mantenimiento lo
    cual reducirá en gran medida el costo de operación y
    actualización.~\cite{Pedrycz2017}
    \item{\textbf{Tecnología de vanguardia:}} El sistema será desplegado en una
    solución de \hyperlink{abbr}{SE} optimizada para su uso en aplicaciones de
    \hyperlink{abbr}{IA}. 
    \item{\textbf{Capacidad de expansión:}} El proyecto de desarrollo del \hyperlink{abbr}{SDAC}
    siempre tendrá como enfoque final la capacidad de expandir y mejorar el
    rendimiento del sistema. Se dejarán todas las disposiciones para mejorar la
    funcionalidad del software con la creación posterior de un sistema que
    integre todos los dispositivos desplegados en el campo para crear una red
    que capturará y procesará conocimiento o añadir nuevos modelos,
    arquitecturas o algoritmos.
\end{itemize}
