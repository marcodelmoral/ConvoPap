\subsection{Objetivo General}

Desarrollar el diseño conceptual de un Sistema de Diagnóstico Asistido por
Computadora (\hyperlink{abbr}{SDAC}\nomenclature{SDAC}{Sistema de Diagnóstico
Asistido por Computadora}) que integre hardware y de software para asistir al
experto cito-tecnólogo en la identificación de células atípicas dentro de una
muestra de Papanicolau observada bajo microscopio usando Deep Learning y
desplegado en un Sistemas Embebido para su uso dentro del laboratorio con el
objetivo de reducir la tasa de falsos positivos y negativos en el diagnóstico..

\subsection{Objetivos Específicos}

\begin{enumerate}
    \item Recolectar una Base de Datos \footnote{En Ciencia de Datos, a las Bases de Datos se les conoce como datasets o conjuntos de datos} (\hyperlink{abbr}{BD}\nomenclature{BD}{Base
    de Datos}) previamente validada y etiquetada para maximizar el rendimiento
    del SDAC.
    \item Explorar y analizar la BD para determinar el número de clases, tamaño
    de las imágenes, tipo de archivo, etcétera. 
    \item Procesar la base de datos mediante algoritmos de Procesamiento Digital
    de Imágenes (\hyperlink{abbr}{PDI}\nomenclature{PDI}{Procesamiento Digital
    de Imágenes}) para obtener imágenes variadas y representativas para mejorar
    la capacidad de generalización.
    \item Realizar experimentos para determinar, dentro de un conjunto de
    arquitecturas, cual es la mejor de todas para el resolver el problema.
    \item Entrenar, validar y analizar el modelo resultante para evaluar
    rendimiento y comprobar supuestos metodológicos.
    \item Preparación del \hyperlink{abbr}{SE} e implementación del modelo
    entrenado en un sistema de \emph{software} con capacidad de capturar
    imágenes de microscopio y de interactuar con el usuario mediante una
    interfaz gráfica táctil.
    % \item Despliegue del sistema dentro del dispositivo de \emph{hardware} y
    % configuración del \emph{software} para su uso
\end{enumerate}