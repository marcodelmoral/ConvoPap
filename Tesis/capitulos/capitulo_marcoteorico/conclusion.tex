El cáncer es una de las batallas más cruentas que libra el hombre. Es
literalmente combatir a nuestro propio cuerpo, a veces en suma desventaja.
Características del cáncer lo hacen especialmente letal, como que genera sus
propios vasos sanguíneos para alimentarse a costa de la persona que lo padece.
Es por ello que debemos incrementar los esfuerzos no solo para erradicarlo, sino
especialmente para detectarlo a tiempo. Es una batalla tan cruel que el
tratamiento se reduce a intoxicar el cuerpo del paciente hasta que uno de los
dos, el cáncer o el paciente, se rinda primero.

Es por ello que se propone el uso de la tecnología más nueva para atacar el
problema del diagnóstico de \hyperlink{abbr}{CCU}. Estas \hyperlink{abbr}{ConvNet}s están alcanzando rendimiento
supra-humano en lugares y situaciones que anteriormente jamás se habría pensado
que eran propensas a automatizarse. Validar correctamente estos modelos es lo
más importante ya que cualquier error puede costar vidas, por ello es que se
usan tantas métricas de evaluación, para asegurar la asertividad del modelo.

La aplicación de \hyperlink{abbr}{DL} es un proceso complejo y computacionalmente intensivo y no
podríamos haber realizado tanto dentro de estas tesis si no hubiese sido por la
empresa Nvidia y su desinteresada donación de un \hyperlink{abbr}{GPU} de última generación.

Asistir a la toma de decisiones directamente con el experto y no a su costa, con
un \hyperlink{abbr}{SDAC}, mejora las condiciones de la tarea a realizar lo cual
incrementa la probabilidad de éxito. El dispositivo asistirá al experto en
aquellos casos fáciles de clasificar, dejándolo enfocarse en los casos donde su
pericia sea requerida. El hardware también ofrece otra ventaja, que mejora la
calidad de vida de la persona que realiza el trabajo, reduciendo la fatiga
ocular y por lo tanto la tasa de error humano.


