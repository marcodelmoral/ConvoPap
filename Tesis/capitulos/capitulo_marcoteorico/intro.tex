En este capítulo se presenta un resumen de las áreas del conocimiento más
significativas para la elaboración de esta tesis. No pretende ser una lista
exhaustiva ya que estas áreas del conocimiento son grandes en número y por si
solas son un mundo completo de investigación. 

Primeramente abordaremos el tema del cáncer ofreciendo una descripción general
de lo que es el cáncer como enfermedad y en que partes del cuerpo se puede
originar, luego viene una breve pero concisa reseña histórica de su diagnóstico
y se mostrará la gran cantidad de causas del cáncer.

Una vez sentadas las bases, se describirá el \hyperlink{abbr}{CCU}, en que parte
se origina, cuales son las causas principales. Su sintomatología y el grado de
peligro que representa. Demostraremos que el cáncer es una enfermedad gradual y
que tiene varias etapas morfológicas que pueden servir para predecir su
desarrollo. La epidemiología y los factores de riesgo del cáncer nos ayudan a
tener estadísticas sobre su incidencia en la población, permitiendo a los
expertos en salud elaborar políticas para reducirla. Se concluye mostrando las
pruebas existentes para diagnosticarlo o descartarlo, así como los criterios
específicos de cada una de ellas que determinan la decisión del experto.

En la sección de la \hyperlink{abbr}{IA}, hablaremos generalmente del concepto
para adentrarnos directamente en el conjunto de técnicas: \hyperlink{abbr}{PDI},
\hyperlink{abbr}{ML} y \hyperlink{abbr}{ConvNet}s que nos permitirán generar el
motor de inferencia para el \hyperlink{abbr}{SDAC}. Haciendo especial énfasis en
las técnicas y formas de evaluar y elegir el mejor modelo.Desmenuzaremos los
componentes de un algoritmo de \hyperlink{abbr}{ConvNet} y enumeraremos los
pasos necesarios para convertirlo en un clasificador poderoso. También se
tratarán los pormenores de las arquitecturas elegidas y el equipo requerido para
el entrenamiento.