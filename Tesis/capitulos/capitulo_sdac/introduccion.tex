Desde hace más de 10 años, la patología digital dentro de la industria ha sido
considerada como uno de los temas más importantes y fascinantes del campo. Sin
embargo, en la actualidad hay muy pocos laboratorios que cuentan con equipo
digital en su flujo de trabajo. Las razones generalmente son legales o
financieras. Los países ejercen poder regulatorio para evitar sistemas
totalmente automatizados.

El \hyperlink{abbr}{SDAC} descrito en esta sección será de bajo costo,
constituido de software libre y armado con métodos de manufactura pertenecientes
a la industria 4.0 que son económicos y accesibles. El sistema no es totalmente
automatizado, trabaja en conjunto con el cito-tecnólogo para holísticamente
detectar y clasificar células atípicas.

Después de introducir la metodología básica. Se desglosarán las actividades
realizadas en cada uno de ellos para todos los experimentos realizados en esta
tesis y se darán ejemplos de su uso en otras aplicaciones de
\hyperlink{abbr}{ML} para complementar.

Las fases de exploración, pre-procesamiento, elección algorítmica y ajuste del
modelo están completamente documentadas en sus respectivos \emph{Jupyter
Notebook} como evidencia del trabajo realizado así como para cumplir con los
requerimientos experimentales de replicación y reproductibilidad. Estos serán
alojados en el repositorio remoto dentro del sitio
\hyperlink{https://github.com/marcojulioarg/ConvoPap}{Github/MarcoJulio} donde
están libres para ser descargados y usados para investigación y como referencia
de esta tesis.

El diseño conceptual culmina con la integración del modelo dentro de un
prototipo de hardware basado en \hyperlink{abbr}{SE}s. Se utilizará la impresión
3D para construir las partes protectoras y adaptadores para el correcto
funcionamiento del dispositivo, aunque por restricciones de tiempo ya que la
impresión es muy tardada, en algunos casos solo se presentará el diseño o un
prototipo simple.