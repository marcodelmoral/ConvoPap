\subsection{Despliegue, monitoreo y mantenimiento del sistema}

La última fase es multidisciplinaria, se trabajará en conjunto con expertos en
el área de sistemas para integrarla a los sistemas de producción de la
organización, ellos nos auxiliarán en la parte técnica mientras que nosotros nos
enfocaremos en el trabajo de monitoreo y métricas de mantenimiento del sistema.

\begin{enumerate}
    \item Preparar la solución para su despliegue a producción: Eliminar código
    basura, preparar entrada de datos reales y escribir pruebas unitarias.
    \item Escribir código de monitoreo en tiempo real: El rendimiento de los
    modelos tiene a degradarse con el tiempo, por ende es necesario monitorearlo
    en tiempo real para detectar cuando el modelo sale de los valores de
    tolerancia requeridos.
    \item Reentrenar el modelo con nuevos datos: Para evitar que se degrade el
    modelo, es necesario reentrenarlo periodicamente y mantener sus niveles de
    acertividad al máximo.
\end{enumerate}

Esta última fase tiene la característica de ser contínua, es decir, una vez
encontrado el mejor modelo, entrenarlo con los mejores datos y optimizado para
el mejor rendimiento, siempre necesitará mantenimiento durante su uso. 

La mejor carta de presentación de la solución será su rendimiento en tiempo real
dentro del laboratorio. Por lo que se pretende aplicar las mejores técnicas de
UX (User eXperience) para reducir el estrés inherente al uso de dispositivos
tecnológicos y maximizar su usabilidad. Esto se explicará en su capítulo
correspondiente.

\subsection{Despliegue, monitoreo y mantenimiento del sistema}

Su despliegue se realizará dentro de un sistema embebido, por lo que se tiene
que poner especial atención a la optimización del modelo para poder ser
ejecutado en entornos con limitado poder computacional. El monitoreo y
reentrenamiento constantes no están contemplados en este trabajo, pero se tienen
presentes en el desarrollo del mismo.