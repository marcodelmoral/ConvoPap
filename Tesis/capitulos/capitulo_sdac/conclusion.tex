El fluir de la propuesta a través de la metodología la transformó en cada etapa
en algo nuevo, acercándolo cada vez más de una idea abstracta a algo material.
Cada etapa por si sola es un mar de conocimiento ya que el conjunto de problemas
existentes a los que se le puede aplicar \hyperlink{abbr}{IA} es infinito. El
análisis general es la parte más importante del proceso, este trabajo demostró
que un marco teórico bien fundamentado más un estado del arte bien elaborado
puede convertir una problemática en una propuesta bien delimitada con altas
posibilidades de transformación.

Gracias al apoyo de Nvidia y la Universidad del Mar Egeo se logró culminar la
metodología, sin ese apoyo, es probable que este trabajo se tratase de algo más
modesto. Las restricciones de tiempo fueron un factor determinante en este
capítulo, no solo el entrenamiento de los modelos en cada uno de los tres
experimentos fue sumamente tardado, sino que también la falta de tiempo impactó
en la fase de implementación al restringir el número de piezas que se pueden
imprimir. La propuesta pudo ser convertida en datos rápidamente sirviendo de
atajo a la fases posteriores, en condiciones normales y como pudimos ver en las
tesis asociadas a la \hyperlink{abbr}{BD}, obtener datos puede llegar a ser
tarea de una tesis entera.

Explorar la \hyperlink{abbr}{BD} fue un proceso sencillo y rápido, para datos
tabulares esto puede llegar a ser más tedioso y requerir algoritmos más
elaborados. Incluida en esta información están las transformaciones usadas
para pre-procesar y aumentar la \hyperlink{abbr}{BD}, enriqueciendo la información
para extraer conocimiento, es ahora cuando puede comenzar el aprendizaje.

Una vez aumentada la información en conocimiento, pudimos comenzar la búsqueda
del mejor algoritmo. Esta fase es inherentemente computacionalmente intensiva,
ya que por restricciones teóricas, es imposible saber cual es la mejor
arquitectura. Ya con la mejor se procedió a diseñar los tres experimentos
principales.

Sumando el tiempo transcurrido en todos los experimentos, se logra dimensionar
el tiempo necesario para conseguir un buen modelo. Considerando que se está
trabajando con un sistema de cómputo de vanguardia con el mejor hardware
especializado para la tarea; hace ver la razón por la cual la integración del
\hyperlink{abbr}{DL} más allá de las grandes corporaciones es lenta. Esto es
lamentable ya que se demostró que puede alcanzar un rendimiento excepcional.

Para finalizar el diseño conceptual y poder convertir el modelo en un sistema
que aspire a interactuar con el usuario final. Se trabajó en acoplar software y
hardware en un solo punto: el \hyperlink{abbr}{SDAC}. Que usará el modelo para
clasificar células dentro de un \hyperlink{abbr}{SE}.