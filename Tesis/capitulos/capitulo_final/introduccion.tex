La creación del \hyperlink{abbr}{SDAC} fue un proceso si bien directo de la
propuesta al sistema, largo gracias a la complejidad computacional del modelo
que realizará la inferencia. Como se expuso, previamente los sistemas de apoyo
al diagnóstico eran caros y poco presentes dentro del laboratorio; al crear un
sistema diseñado desde el concepto para maximizar su presencia y su uso por el
experto, se asegura que pueda reducir las muertes por \hyperlink{abbr}{CCU} en
nuestro país gracias a la detección rápida y temprana de las lesiones
morfológicas que lo originan. 

Concluimos esta tesis presentando un condensado de los resultados de los
experimentos realizados, una reseña del trabajo a futuro necesario para
convertir la solución en un sistema y cerramos con unas conclusiones pertinentes
y algunos comentarios del desarrollo completo de la tesis.