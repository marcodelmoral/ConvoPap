El cáncer cérvico-uterino es un problema de salud latente a nivel mundial. Los
esfuerzos para reducir su incidencia son asimétricos en el aspecto en que si
bien cada día se realizan más pruebas, no se conjuga con un entrenamiento de más
expertos para satisfacer la demanda que el incremento de pruebas genera.

Detectar a tiempo el cáncer es la primera línea de defensa y es la variable que
más importa en la predicción de la supervivencia, no solo para el
\hyperlink{abbr}{CCU} sino para todos los tipos de cáncer.

Poder clasificar correctamente entre células normales y anormales a veces no
es suficiente, también se requiere determinar el grado de lesión para que el
médico pueda establecer horizontes temporales para realizar un mejor
tratamiento, es por ello que tomó especial énfasis en el experimento de
clasificación multi-clase.

En las aplicaciones de modelos de clasificación al área médica, se debe de poner
especial atención al rendimiento del algoritmo y su evaluación para conocer como
se comporta en el problema determinado. Un algoritmo aplicado a diagnóstico no
puede cometer errores como clasificar una célula sana con una enferma, pero es
mucho mayor problema si clasifica una célula enferma como una sana, en ese caso,
puede morir gente. Medir el precisamente el rendimiento del modelo, tanto la
estimación real del error de generalización así como las métricas de
clasificación, nos da la certeza necesaria para determinar si sirve para el
diagnóstico médico o no.

El \hyperlink{abbr}{DL} en medicina se ha topado con cierta renuencia debido a
que la mayoría de los modelos se comportan como caja negra y carecen de
interpretabilidad, sobre todo si se comparan con algoritmos como los Árboles de
Decisión que son totalmente transparentes. Es por ello que se tomó especial
atención en analizar que es lo que ve la \hyperlink{abbr}{ConvNet} al clasificar
una imagen y poder primeramente comprobar los supuestos metodológicos de
investigación y segundo, saber si la red está observando la parte correcta de la
imagen y hacerla más interpretable.

Si bien el paradigma actual de los \hyperlink{abbr}{SDAC}s es la automatización
total del proceso de diagnóstico, el sistema no reemplaza al experto sino que
está presente para asistirlo en la clasificación de células fáciles de
diagnosticar, dejando el camino libre para que se enfoque exclusivamente en
aquellas células que por su morfología presentan ambigüedad en su clasificación;
con el objetivo de diagnosticar más muestras por jornada de trabajo, para tener
mejor rendimiento en su clasificación y tener un incremento en la ergonomía del
trabajo al ofrecer una pantalla en lugar del objetivo pequeño del microscopio,
que fuerza al ojo y degrada la vista del experto a largo plazo, y a corto, lo
fatiga reduciendo su exactitud.

El proceso de convertir una propuesta en un sistema mediante la metodología
propuesta orquestado directamente por las ganas de que el trabajo expuesto en
esta tesis tenga un respaldo tanto en código como en resultados. El repositorio
remoto incluido en la tesis es un acompañante y complemento del desarrollo aquí
expuesto. Se recomienda sobremanera su análisis al leer esta tesis.

Usar \hyperlink{abbr}{DL} simplifica mucho la metodología general de
\hyperlink{abbr}{ML}. Las dos fases más difíciles de todo el proceso es sacar
las características importantes y decisoras dentro de la imagen, usando
algoritmos complejos y convertir todo en datos tabulares. Se dieron ejemplos de
las fases para el uso en algoritmos más tradicionales para contrastar las
ventajas del diseño conceptual propuesto.

Gracias a la \hyperlink{abbr}{BD} del Hospital de Herlev, se pudo realizar un
pase rápido por las cuatro primeras fases del proyecto. Los investigadores
mantuvieron una comunicación expedita con el autor y resolvieron todas las dudas
de manera precisa y amable. 

La etapa de \hyperlink{abbr}{DL} fue especialmente compleja. Las restricciones
computacionales (aliviadas en gran parte por la donación de Nvidia) fueron
importantes. Para poder llevar a cabo los entrenamiento, se adquirió un equipo
de cómputo de última generación. Esto evidencia uno de los contras más comunes
del \hyperlink{abbr}{DL}, las restricciones computacionales. Afortunadamente la
mayoría está confinada a la parte de entrenamiento, pudiendo encontrar una
plataforma especializada, de bajo costo y embebida para el despliegue del
modelo.

Debido a las nuevas técnicas de manufactura 4.0 como la impresión 3D, a que los
\hyperlink{abbr}{SE} ya vienen integrados en soluciones discretas con los
puertos necesarios para su implementación y a los modernos frameworks escritos
en lenguaje de alto nivel, legible, de código abierto y poderoso se logró
realizar un diseño conceptual bastante maduro y rápido. Poder realizar un
prototipado rápido y consistente es crucial para poder generar soluciones
acordes a las necesidades del usuario final, en este caso, un experto
cito-tecnólogo.

Configurar la plataforma e instalar las dependencias para el sistema fue un
proceso tedioso y tardado, se espera que a futuro, cuando la tecnología esté
mejor adoptada, se desarrollen comandos para la instalación sencillos y
transparentes. 

El trabajo aquí desarrollado no solo culminó en la creación de un modelo con
mejor rendimiento que el benchmark, sino que tal rendimiento es muy superior a
todas las alternativas antes aplicadas al problema y sus métricas cumplen las
más estrictas estipulaciones médicas. Adicionalmente, se aplicó un método jamás
antes intentado con la \hyperlink{abbr}{BD} de Herlev, la segmentación
semántica; obteniendo resultados interesantes cuando se probó la misma red en
laminillas de campo abierto. 

Si bien la idea de un microscopio inteligente no es nueva, la aplicación de este
concepto con el de un \hyperlink{abbr}{SDAC} como el desarrollado en esta tesis,
puede traer beneficios a la población, a incrementar la calidad del trabajo del
experto y reducir significativamente la cantidad de casos de
\hyperlink{abbr}{CCU} que están a la espera de ser diagnosticados.

El diseño conceptual final es innovador en el aspecto de que no solo conjuga las
mejores tecnologías con los algoritmos más modernos, ni por la cantidad superior
de pruebas de análisis de clasificación y de comportamiento neuronal, sino que
por las características en su integración y posterior despliegue, su enfoque a
asistir a la toma de decisiones en lugar de automatizarlas y por la posibilidad
de mejorar el rendimiento no de un proceso sino de un experto humano, rompe con
todos los paradigmas previos asociados tanto en \hyperlink{abbr}{DL},
\hyperlink{abbr}{SDAC}s y en la \hyperlink{abbr}{IA} de la actualidad.